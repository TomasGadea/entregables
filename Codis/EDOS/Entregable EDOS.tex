\documentclass[12pt, a4papre]{article}
\usepackage[catalan]{babel}
\usepackage[unicode]{hyperref}
\usepackage{amsmath}
\usepackage{amssymb}
\usepackage{amsthm}
\usepackage{xifthen}

\newcommand{\norm}[1]{\lvert #1 \rvert}

\hypersetup{
    colorlinks = true,
    linkcolor = blue
}

\author{Daniel Vilardell}
\title{Entregable EDOS}
\date{}

\begin{document}
	\maketitle
	\section{Exercici 4}
	\textbf{d)} L'enunciat ens dona el seguent problema de valors inicials
	\[
		y''-2y'+y=te^t; \textrm{  }y(0)=1,y'(0)=0
	\]
	
	Usant les propietats de la transformada de Laplace que ens diuen que 
	\[
		\mathcal{L}\{f'(t)\}=sF(s)-f(0), \mathcal{L}\{f''(t)\}=s^2F(s)-sf(0)-f'(0)
	\]
	
	Trovem la tranformada de la equació donada i veiem que
	\[
		s^2Y(s)-sy(0)-y'(0)-2(sY(s)-y(0))+Y(s)=\frac{1}{(s-1)^2}
	\]
	\[
		Y(s)(s^2-2s+1)-s+2=\frac{1}{(s-1)^2}
	\]
	\[
	Y(s)=\frac{(s-2)(s-1)^2+1}{(s-1)^4}=\frac{s}{(s-1)^2}-\frac{2}{(s-1)^2}+\frac{1}{(s-1)^4}
	\]
	
	Si descomponem $\frac{s}{(s-1)^2}$ en fraccions simples tenim el seguent
	\[
		Y(s)=\left(\frac{1}{(s-1)} + \frac{1}{(s-1)^2}\right)-\frac{2}{(s-1)^2}+\frac{1}{(s-1)^4}
	\]
	\[
		Y(s)=\frac{1}{(s-1)} - \frac{1}{(s-1)^2}+\frac{1}{(s-1)^4}
	\]
	
	I calculem la transformada inversa del resultat trobat que serà la funció en questió
	\[
		\mathcal{L}^{-1}\{Y(s)\}(t)=y(t)=e^t-te^t+\frac{t^3e^t}{3!}
	\]
	\[
		y(t)=e^t\left(\frac{t^3}{6}-t+1\right)
	\]
	
	
	
	
	\newpage
	\section{Exercici 11}
	\textbf{a)} L'enunciat ens dona un altre problema de valors inicials que diu el seguent
	\[
		x''+4x=2\delta(t-\pi); \textrm{  }x(0)=1, x'(0)=0
	\]
	
	Usant la mateixa propietat que en l'exercici 4 fet anteriorment tenim que
	\[
		s^2X(s)-sx(0)-x'(0)+4X(s)=2e^{-\pi s}
	\]
	\[
		X(s)=\frac{2e^{-\pi s}+s}{s^2+4}=2e^{-\pi s}\frac{2}{s^2+4}+\frac{s}{s^2+4}
	\]
	
	Tenint en conte que $\mathcal{L}^{-1}\{e^{-as}F(s)\}=f(t-a)u(t-a)$ i coneixent les transformades conegudes de $\sin$ i de $\cos$ tenim que
	\[
		\mathcal{L}^{-1}\{X(s)\}(t)=x(t)=sin(2(t-\pi))u(t-\pi)+cos(2t)
	\]
	
	Com que el temps en tal exercici no pot ser negatiu el resultat dona el següent
	\[
		x(t)=(sin(2t)u(t-\pi)+cos(2t))u(t)
	\]
	
\end{document}