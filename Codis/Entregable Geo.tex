\documentclass[12pt, a4papre]{article}
\usepackage[catalan]{babel}
\usepackage[unicode]{hyperref}
\usepackage{amsmath}
\usepackage{amssymb}
\usepackage{amsthm}
\usepackage{xifthen}

\newcommand{\norm}[1]{\lvert #1 \rvert}

\hypersetup{
    colorlinks = true,
    linkcolor = blue
}

\author{Daniel Vilardell}
\title{Entregable Geometria Afí i Euclidea}
\date{}

\begin{document}
	\maketitle
	\section{Problema 1}
	\textbf{(i)} 
	\begin{proof}
		 Tenint en conta que es un gir al voltant d'un punt, podem considerar la matriu de \textit{F} en la referencia $R=\{p; u_1, u_2\}$ on 
		$p$ es el punt de rotació i $u_1$ i $u_2$ son vectors arvitraris ortonormals, com a
		\[
		M(F)= 
		\begin{pmatrix}
			c\alpha &	- s\alpha	& 0\\
			s\alpha  & c\alpha	& 0\\
			0    & 	0	& 1 
		\end{pmatrix}
		\]
		Per comoditat en els calculs prendrem $u_1=(q_1-p)$ i per tant en aquesta referencia $R$ el punt $q=(1,0)$ i per tant $q_2$ sera de la forma
		\[
		q_2=
		\begin{pmatrix}
			c\alpha &	- s\alpha	& 0\\
			s\alpha  & c\alpha	& 0\\
			0    & 	0	& 1 
		\end{pmatrix}
		\begin{pmatrix}
			1\\
			0 \\
			1 \\
		\end{pmatrix}
		=
		\begin{pmatrix}
			c\alpha\\
			s\alpha \\
			1 \\
		\end{pmatrix}
		\]
		I per tant $u=q_2-q_1=(c\alpha-1,s\alpha)$. Per a veure el que ens demanen considerarem el vector $s=(\frac{q_1-q_2}{2}-p)=(\frac{1+c\alpha}{2},\frac{s\alpha}{2})$ 
		i veurem que es perpendicular ja que $\langle s, u\rangle = 0$.
		\[
		\langle s, u\rangle = (\frac{1+c\alpha}{2},\frac{s\alpha}{2})\cdot(c\alpha-1,s\alpha)=
		\frac{c\alpha+1}{2}(c\alpha-1)+\frac{s^2\alpha}{2}=\frac{s^2\alpha+c^2\alpha-1}{2}=0
		\]
		$\implies$ El punt $p$ pertany la recta perpendicular a $u$ que passa pel punt $s$.
		
	\end{proof}
	\newpage
	Per a trobar el centre $p$ del gir usarem l'apartat anterior  que afirma que la recta perpendicular al vector $u=F(q)-1$ i que passa pel punt
	central entre $F(q)$ i $q$ conte el centre del gir. Primer de tot trobarem aquesta recta per a els dos punts i les seves imatges que ens han donat
	i en el cas que aquestes siguin linealment independents podrem trobar la seva interseccio. Com que les dos rectes hauràn de contenir el punt
	$p$ aquesta intersecció serà el centre del gir.
	
	Tenim que $\frac{F(1,1)-(-1,3)}{2}$ forma part de la recta i que el seu vector director $s_1$ es $s|\langle s_1, u\rangle=0$ on u es el vector $u=(-1,3)-F(1,1)=(-2-2)$
	Es veu que aquest vector ha de ser $s_1=(1,1)$. Per tant la recta es $r_1=(0,2)+[(1,1)]$. 
	
	Analogament, si fem el mateix amb els altres dos punts obtenim la recta $r_2=(1,2)+[(2,1)]$. I per tant el punt $p$ del centre sera
	
	\[
	p=r_1\cap r_2 = (-1,1)
	\]
	Per a trobar el angle de gir buscarem el angle entre els vectors $(2,0)-p$ i $F(2,0)-p=(0,4)-p$, es a dir $(1,3)$ i $(3,-1)$. Tenim que pel que hem vist a teoria que
	\[
	cos\alpha=\frac{(1,3)\cdot(3,-1)}{\norm{(1,3)}\norm{(3,-1)}}=0 \implies \alpha=\pm\frac{\pi}{2} 
	\]
	Per a trobar el sentit farem el determinant de la matriu dels vectors.
	\[
	\begin{vmatrix}
			3 &	1\\
			-1  & 3\\
	\end{vmatrix}
	=10 > 0 \implies \alpha=\frac{\pi}{2}
	\]
	Finalment podem trobar $M(F)$ tenint en conta que sabem que la imatge del centre es el centre podem trobar la matriu buscada.
	\[
	\begin{pmatrix}
		0 &	-1	& a\\
		1  & 0	& b\\
		0    & 0	& 1 
	\end{pmatrix}
	\begin{pmatrix}
		-1\\
		1\\
		1 \\
	\end{pmatrix}
	=
	\begin{pmatrix}
		-1\\
		1\\
		1 \\
	\end{pmatrix}
	\iff
	a=0, b=2
	\]
	I per tant
	\[
	M(F)=
	\begin{pmatrix}
		0 &	-1	& 0\\
		1  & 0	& 2\\
		0    & 0	& 1 
	\end{pmatrix}
	\]
	\newpage
	\textbf{(ii)} 
	\begin{proof}
	
		 Tenint en conta que es un gir al voltant d'una recta, podem considerar la matriu de \textit{F} en la referencia $R=\{p; u_1, u_2, u_3\}$ on 
		$p$ es un punt de la recta de punts fixos, $u_1$ es el vector director d'aquesta recta de punts fixoss i $u_2$ i $u_3$ son vectors 
		arvitraris ortonormals entre ells i a $u_1$, com a
		\[
		M(F)= 
		\begin{pmatrix}
			1	&	0	&	0	&	0\\
			0	&	c\alpha &	- s\alpha	& 0\\
			0	&	s\alpha  & c\alpha	& 0\\
			0	&	0    & 	0	& 1 
		\end{pmatrix}
		\]
		
		Per comoditat en els calculs prendrem $u_2=(q_1-p)$ i per tant en aquesta referencia $R$ el punt $q=(0, 1, 0)$ i per tant $q_2$ sera de la forma
		\[
		q_2=
		\begin{pmatrix}
			1	&	0	&	0	&	0\\
			0	&	c\alpha &	- s\alpha	& 0\\
			0	&	s\alpha  & c\alpha	& 0\\
			0	&	0    & 	0	& 1 
		\end{pmatrix}
		\begin{pmatrix}
			0\\
			1\\
			0 \\
			1 \\
		\end{pmatrix}
		=
		\begin{pmatrix}
			0\\
			c\alpha\\
			s\alpha \\
			1 \\
		\end{pmatrix}
		\]
		
		I per tant $u=q_2-q_1=(0, c\alpha-1,s\alpha)$. Per a veure el que ens demanen considerarem el vector $s_1=(\frac{q_1-q_2}{2}-p)=(0, \frac{1+c\alpha}{2},\frac{s\alpha}{2})$ 
		i el vector director de la recta de punts fixos $s_2=(1,0,0)$ i veurem que son perpendiculars a $u$ ja que $\langle s_i, u\rangle = 0, i=1,2$.
		\[
		\langle s_1, u\rangle =(1,0,0)\cdot(0, c\alpha-1,s\alpha)=0
		\]
		\[
		\langle s_2, u\rangle = (0, \frac{1+c\alpha}{2},\frac{s\alpha}{2})\cdot(0, c\alpha-1,s\alpha)=
		\frac{c\alpha+1}{2}(c\alpha-1)+\frac{s^2\alpha}{2}=\frac{s^2\alpha+c^2\alpha-1}{2}=0
		\]
		
		I per tant la recta de punts fixos respecte la qual es fa la rotació esta continguda dins del pla perpendicular a $u$ que conté el punt $(p+q)/2$.
		
	\end{proof}
	\newpage
	
	Per a trobar en primer lloc la recta de punts fixos aplicarem el mateix procediment que en 2 dimensions en l'apartat (i), es a dir, trobar els plans perpendiculars als respectius vectors
	$u$ i trobar la seva intersecció, que aquesta per el mateix argument fet anteriorment es la recta de punts fixos buscada.
	
	Tenim que els vectors directors del primer pla son 
	\[
	\{((1,1,1)-F(1,1,1))\}^\perp=\{(0,0,1)\}^\perp=\{(1,0,0),(0,1,0)\}
	\]
	
	I el punt $\frac{(F(1,1,1)+(1,1,1))}{2}=(1,1,\frac{1}{2})$, per tant el pla buscat es el 
	\[
	\Pi_1=(1,1,\frac{1}{2}) + [(1,0,0),(0,1,0)]
	\]
	
	Analogament fent el mateix procediment per al segon pla tenim que
	\[
	\{((0,1,0)-F(0,1,0))\}^\perp=\{(1,-1,1)\}^\perp=\{(1,1,0),(0,1,1)\}
	\]
	I el punt $\frac{(F(0,1,0)+(0,1,0))}{2}=(\frac{1}{2},\frac{1}{2},\frac{1}{2})$, per tant el segon pla buscat es el 
	\[
	\Pi_2=(\frac{1}{2},\frac{1}{2},\frac{1}{2}) + [(1,1,0),(0,1,1)]
	\]
	
	Es pot veure facilment que un punt d'intersecció entre els dos plans es el $(1,1,\frac{1}{2})$ i també que $\{(1,0,0),(0,1,0)\}\cap \{(1,1,0),(0,1,1)\}=\{(1,1,0)\}$, per tant
	la recta de punts fixos es
	\[
	r=(1,1,\frac{1}{2})+[(1,1,0)]
	\]
	
	Ara que tenim la recta de punts fixos podem veure clarament que conte els punts mitjos entre els punts donats i les seves imatges. L'unica rotació en la que aixo passa
	es aquella amb $\alpha=\pi$, i per tant aquest es el angle de rotació.
	
	Finalment per a trobar les equacions considerarem la matriu de la aplicació en la referencia $R=\{(0,0,\frac{1}{2}); (\frac{1}{\sqrt{2}}, \frac{1}{\sqrt{2}} 0), (0,0,1), (\frac{1}{\sqrt{2}} , -\frac{1}{\sqrt{2}} , 0)\}$
	on el centre es un punt fix de la recta respecte la que es fa el gir, i el primer vector de la base es el vector director d'aquesta recta. Comprovem primer que la base es positiva.
	\[
	\begin{vmatrix}
		\frac{1}{\sqrt{2}} 	&	0	&	\frac{1}{\sqrt{2}}\\
		\frac{1}{\sqrt{2}} 	&	0	&	-\frac{1}{\sqrt{2}}\\
		0				&	1  	&	1\\
	\end{vmatrix}
	= 1 > 0
	\]
	
	I per tant es una base positiva. En aquesta referencia la matriu te la forma
	
	\[
	M(F)_R=
	\begin{pmatrix}
		1	&	0		&	0		&	0\\
		0	&	cos(\pi)	&	sin(\pi)	&	0\\
		0	&	sin(\pi) 	&	cos(\pi)	&	0\\
		0	&	0		& 	0		&	1
	\end{pmatrix}
	= 
	\begin{pmatrix}
		1	&	0	&	0	&	0\\
		0	&	-1	&	0	&	0\\
		0	&	0  	&	-1	&	0\\
		0	&	0	& 	0	&	1
	\end{pmatrix}
	\]
	
	Finalment tenim que 
	
	\[
	M(F)=S\cdot M(F)_R\cdot S^{-1}=
	\]
	\[
	\begin{pmatrix}
		\frac{1}{\sqrt{2}} 	&	0	&	\frac{1}{\sqrt{2}} 	&	0\\
		\frac{1}{\sqrt{2}} 	&	0	&	-\frac{1}{\sqrt{2}}	&	0\\
		0				&	1  	&	0				&	\frac{1}{2}\\
		0				&	0	& 	0				&	1
	\end{pmatrix}
	\begin{pmatrix}
		1	&	0	&	0	&	0\\
		0	&	-1	&	0	&	0\\
		0	&	0  	&	-1	&	0\\
		0	&	0	& 	0	&	1
	\end{pmatrix}
	\begin{pmatrix}
		\frac{1}{\sqrt{2}} 	&	0	&	\frac{1}{\sqrt{2}} 	&	0\\
		\frac{1}{\sqrt{2}} 	&	0	&	-\frac{1}{\sqrt{2}}	&	0\\
		0				&	1  	&	0				&	\frac{1}{2}\\
		0				&	0	& 	0				&	1
	\end{pmatrix}^{-1}
	\]
	
	Per a trobar tal inversa usarem la formula que ens diu que la part lineal de la matriu $S^{-1}$ es la inversa de la part lineal de $S$
	if que la part no lineal es $S^{-1}$ multiplicat per la part no lineal de $S$.
	\[
	\begin{pmatrix}
		\frac{1}{\sqrt{2}} 	&	0	&	\frac{1}{\sqrt{2}}  & 1 & 0 & 0\\
		\frac{1}{\sqrt{2}} 	&	0	&	-\frac{1}{\sqrt{2}} & 0 & 1 & 0\\
		0				&	1  	&	0			  & 0 & 0 & 1\\
	\end{pmatrix}
	\sim
	\begin{pmatrix}
		\frac{1}{\sqrt{2}} 	&	0	&	\frac{1}{\sqrt{2}}  & 1 & 0 & 0\\
		0			 	&	1	&	0		 	  & 0 & 0 & 1\\
		0				&	0  	&	-\sqrt{2}		  & -1 & 1& 0\\
	\end{pmatrix}
	\sim
	\begin{pmatrix}
		1 	&	0	&	0 	& \frac{1}{\sqrt{2}}	 & \frac{1}{\sqrt{2}}	 & 0\\
		0	&	1	&	0	& 0 				& 0 				& 1\\
		0	&	0  	&	1	& \frac{1}{\sqrt{2}}	& -\frac{1}{\sqrt{2}}	& 0\\
	\end{pmatrix}
	\]
	
	I per tant la inversa buscada es
	\[
	S^{-1}=
	\begin{pmatrix}
		\frac{1}{\sqrt{2}}	 & \frac{1}{\sqrt{2}}	 & 0\\
		0 				& 0 				& 1\\
		\frac{1}{\sqrt{2}}	& -\frac{1}{\sqrt{2}}	& 0\\
	\end{pmatrix}
	\]
	
	La part no lineal la trobem fent la operacio mencionada abans
	
	\[
	-
	\begin{pmatrix}
		\frac{1}{\sqrt{2}}	 & \frac{1}{\sqrt{2}}	 & 0\\
		0 				& 0 				& 1\\
		\frac{1}{\sqrt{2}}	& -\frac{1}{\sqrt{2}}	& 0\\
	\end{pmatrix}
	\begin{pmatrix}
		0\\
		0\\
		\frac{1}{2}\\
	\end{pmatrix}
	=
	\begin{pmatrix}
		0\\
		-\frac{1}{2}\\
		0\\
	\end{pmatrix}
	\]
	
	\newpage
	Finalment calculem $M(F)$
	\[
	M(F)=
	\begin{pmatrix}
		\frac{1}{\sqrt{2}} 	&	0	&	\frac{1}{\sqrt{2}} 	&	0\\
		\frac{1}{\sqrt{2}} 	&	0	&	-\frac{1}{\sqrt{2}}	&	0\\
		0				&	1  	&	0				&	\frac{1}{2}\\
		0				&	0	& 	0				&	1
	\end{pmatrix}
	\begin{pmatrix}
		1	&	0	&	0	&	0\\
		0	&	-1	&	0	&	0\\
		0	&	0  	&	-1	&	0\\
		0	&	0	& 	0	&	1
	\end{pmatrix}
	\begin{pmatrix}
		\frac{1}{\sqrt{2}}	 & \frac{1}{\sqrt{2}}	& 0	&0\\
		0 			& 0 				& 1	&-\frac{1}{2}\\
		\frac{1}{\sqrt{2}}	& -\frac{1}{\sqrt{2}}	& 0	&0\\
		0			&0				&0	&0
	\end{pmatrix}
	\]
	\[
	M(F)=
	\begin{pmatrix}
		0	&	1	&	0	&	0\\
		1	&	0	&	0	&	0\\
		0	&	0  	&	-1	&	1\\
		0	&	0	& 	0	&	1
	\end{pmatrix}
	\]
	
	
	
	
	
	\newpage
	\section{Problema 2}
	\textbf{(i)} 
	En primer lloc trobarem $M(f)$ per tal de composarla amb $h$. Les equacions les podem obtenir imposant les condicions $\overrightarrow{qf(q)} \perp \pi$ on $\pi$ es
	el pla respecte el que es fa la simetria especular $\pi=p + [w]^\perp$, $q$ un punt arbitrari. Ho farem usant la seguent formula
	\[
	f(q)=q-2*\frac{\langle pq, w \rangle}{\langle w, w, \rangle} w
	\]
	Tenim que $\pi: (0,0,0)+[(1,-1,0)]^\perp$
	\begin{center}
	\[
	f(x,y,z)=(x,y,z)-2\frac{\langle (x,y,z), (1,-1,0) \rangle}{1^2+(-1)^2+0^2}(1,-1,0)=
	\]
	\[
	(x,y,z)-(x-y,y-x,0)=(y,x,z)
	\]
	\end{center}
	I per tant la matriu M(f) serà de la forma
	\[
	M(f)=
	\begin{pmatrix}
		0	&	1	&	0	&	0\\
		1	&	0	&	0	&	0\\
		0	&	0  	&	1	&	0\\
		0	&	0	& 	0	&	1
	\end{pmatrix}
	\]
	Aleshores la composició de $h\circ f$ tindra la forma
	\[
	M(F)=M(h\circ f)=
	\begin{pmatrix}
		\frac{1}{3}		&	-\frac{2}{3}	&	-\frac{2}{3}	&	2\\
		-\frac{2}{3}	&	-\frac{2}{3}	&	\frac{1}{3}		&	2\\
		-\frac{2}{3}	&	\frac{1}{3}  	&	-\frac{2}{3}	&	2\\
		0			&	0			& 	0			&	1 
	\end{pmatrix}
	\begin{pmatrix}
		0	&	1	&	0	&	0\\
		1	&	0	&	0	&	0\\
		0	&	0  	&	1	&	0\\
		0	&	0	& 	0	&	1
	\end{pmatrix}
	=
	\begin{pmatrix}
		-\frac{2}{3}	&	\frac{1}{3}		&	-\frac{2}{3}	&	2\\
		-\frac{2}{3}	&	-\frac{2}{3}	&	\frac{1}{3}		&	2\\
		\frac{1}{3}		&	-\frac{2}{3}  	&	-\frac{2}{3}	&	2\\
		0			&	0			& 	0			&	1 
	\end{pmatrix}
	\]
	Com que $|M(F)|=-1$ podem dir que $F$ es inversa. Mirem si te punts fixos.
	\[
	\begin{pmatrix}
		-\frac{5}{3}	&	\frac{1}{3}		&	-\frac{2}{3}\\
		-\frac{2}{3}	&	-\frac{5}{3}	&	\frac{1}{3}	\\
		\frac{1}{3}		&	-\frac{2}{3}  	&	-\frac{5}{3}\\
	\end{pmatrix}
	\begin{pmatrix}
		x\\
		z\\
		y\\
	\end{pmatrix}
	=
	\begin{pmatrix}
		-2\\
		-2\\
		-2\\
	\end{pmatrix}
	\iff
	(x,y,z)=(1,1,1)
	\]
	Per tant podem assegurar que $F$ es una composicio de una simetria especular amb una rotació al voltant de una recta que conté $(1,1,1)$
	El angle de rotacio $\alpha$ el podem trobar usant la formula que diu que $2cos\alpha=Tr(F)+1=-1\iff \alpha=\pm\frac{2\pi}{3}$.
	Per a trobar si esta entre $(0,\pi)$ o entre $(\pi,2\pi)$ calcularem el determinant de la matriu formada per el vector director del eix, el vector $(3,0,0)$
	que no forma part del eix l a seva imatge $F(3,0,0)=(-2,-2,1)$.
	
	\[
	\begin{vmatrix}
		1	&	3	&	-2\\
		1	&	0	&	-2\\
		1	&	0  	&	1
	\end{vmatrix}
	=
	-9<0
	\]
	
	I per tant l'angle de rotacio es $\alpha=\frac{4\pi}{3}$.
	
	Finalment trobarem el plà a partir de trobar el vep de vap 1 i considerar el seu ortogonal.
	\[
	Nuc(M(F)-(-1)Id))=Nuc
	\begin{pmatrix}
		\frac{1}{3}		&	\frac{1}{3}		&	-\frac{2}{3}\\
		-\frac{2}{3}	&	\frac{1}{3}		&	\frac{1}{3}	\\
		\frac{1}{3}		&	-\frac{2}{3}  	&	\frac{1}{3}	\\
	\end{pmatrix}
	=
	Nuc
	\begin{pmatrix}
		1	&	0	&	-1\\
		0	&	1	&	-1\\
		0	&	0  	&	0
	\end{pmatrix}
	=[(1,1,1)]
	\]
	I per tant el pla respecte el que es fa la simetria especular es 
	\[
	\Pi=(1,1,1)+[(1,-1,0),(0,1,-1)]
	\]
	I tenint en conta que el vector director de la recta respecte la que es fa la rotació es el perpendicular al pla de la simetria especular 
	ens queda que tal recta es
	\[
	r=(1,1,1) + [(1,1,1)]
	\]
	\\
	

	\textbf{(ii)} 
	Primer de tot 
	\begin{center}
	\[
	g(x,y,z)=(x,y,z)-2\frac{\langle (x,y,z), (1,1,1) \rangle}{1^2+1^2+1^2}(1,1,1)=
	\]
	\[
	(x,y,z)-\frac{2}{3}(x+y+z,x+y+z,z+y+z)=(\frac{1}{3}x-\frac{2}{3}y-\frac{2}{3}z,\frac{2}{3}x-\frac{1}{3}y-\frac{2}{3}z,\frac{2}{3}x-\frac{2}{3}y-\frac{1}{3}z)
	\]
	\end{center}
	I per tant la matriu M(g) serà de la forma
	\[
	M(g)=
	\begin{pmatrix}
		\frac{1}{3}		&	-\frac{2}{3}	&	-\frac{2}{3}	&	0\\
		-\frac{2}{3}	&	\frac{1}{3}		&	-\frac{2}{3}	&	0\\
		-\frac{2}{3}	&	-\frac{2}{3}  	&	\frac{1}{3}		&	0\\
		0			&	0			& 	0			&	1 
	\end{pmatrix}
	\]
	I per tant la matriu M(G) sera de la forma
	\[
	M(G)=M(g\circ F)=
	\begin{pmatrix}
		\frac{1}{3}		&	-\frac{2}{3}	&	-\frac{2}{3}	&	0\\
		-\frac{2}{3}	&	\frac{1}{3}		&	-\frac{2}{3}	&	0\\
		-\frac{2}{3}	&	-\frac{2}{3}  	&	\frac{1}{3}		&	0\\
		0			&	0			& 	0			&	1 
	\end{pmatrix}
	\begin{pmatrix}
		-\frac{2}{3}	&	\frac{1}{3}		&	-\frac{2}{3}	&	2\\
		-\frac{2}{3}	&	-\frac{2}{3}	&	\frac{1}{3}		&	2\\
		\frac{1}{3}		&	-\frac{2}{3}  	&	-\frac{2}{3}	&	2\\
		0			&	0			& 	0			&	1 
	\end{pmatrix}
	=
	\begin{pmatrix}
		0	&	1	&	0	&	-2\\
		0	&	0	&	1	&	-2\\
		1	&	0  	&	0	&	-2\\
		0	&	0	& 	0	&	1
	\end{pmatrix}
	\]
	Com que $det(M(G))=1$ G es un moviment directe i per tant es un moviment helicoidal. Calculant el vep de vap 1 tindrem la direccio de la translacio.
	\[
	Nuc(M(G)-1)Id))=Nuc
	\begin{pmatrix}
		-1	&	1	&	0\\
		0	&	-1	&	1\\
		1	&	0  	&	-1
	\end{pmatrix}
	=
	Nuc
	\begin{pmatrix}
		1	&	0	&	-1\\
		0	&	1	&	-1\\
		0	&	0  	&	0
	\end{pmatrix}
	=
	[(1,1,1)]
	\]
	Per a trobar la recta respecta la que fa el gir elicoidal nomes ens caldra trobar un punt de la recta, es a dir un punt $p=(x,y,z)$ tal que 
	\[
	\begin{pmatrix}
		0	&	1	&	0	&	-2\\
		0	&	0	&	1	&	-2\\
		1	&	0  	&	0	&	-2\\
		0	&	0	& 	0	&	1
	\end{pmatrix}
	\begin{pmatrix}
		x\\
		z\\
		y\\
		1\\
	\end{pmatrix}
	=
	\begin{pmatrix}
		y-2\\
		z-2\\
		x-2\\
		1\\
	\end{pmatrix}
	=
	\begin{pmatrix}
		x\\
		z\\
		y\\
		1\\
	\end{pmatrix}
	+\lambda
	\begin{pmatrix}
		1\\
		1\\
		1\\
		0\\
	\end{pmatrix}
	\]
	Es pot veure que si considerem el punt $p=(0,0,0)$ i $\lambda=-2$ es compleix la igualtat i per tant 
	\[
	r=(0,0,0)+[(1,1,1)]
	\]
	Observem que si calculem $G(G(G(x,y,z)))=(x-6,y-6,z-6)=(x,y,z) - 6*(1,1,1)$ Com que el vector $(1,1,1)$ es el vector director de la recta respecte la que es fa
	el gir podem asegurar que s'ha d'aplicar 3 cops la funcio per a que faci una rotacio de $2\pi$, es a dir una volta a l'eix, per tant a cada volta gira un angle de 
	$\alpha=\frac{2\pi}{3}$
	\\
	
	\textbf{(iii)} 
	Com que s'aplica la rotació 15 podem considerar que el punt girara $15\alpha=10\pi$ 
	vegades respecte l'eix de rotació. Com que sempre es quedara dins del mateix pla i gira $5(2\pi)$ radians que es el mateix que 5 voltes, podem confirmar que tal punt no
	es mourà per efecte de la rotació. Com que s'aplica un nombre parell de vegades la simetria especular si que variarà el punt en questió deixant-lo en el seu simetric respecte el pla
	$\Pi$ i per tant $F^15(0,0,0)=(2,2,2)$.
	
	Com hem vist en l'apartat \textbf{(ii)} cada cop que s'aplica 3 cops la funcio la imatge fa una volta completa a l'eix de rotació i es desplaça en la direccio $-6(1,1,1)$
	per tant si s'aplica 6 cops la composicio de 3 funcions $G$ ens queda que $G^18(x,y,z)=(x,y,z)+6(-6(1,1,1))=(x-36, y-36, z-36)$ que en el cas del punt que ens donen
	dona $G^18(0,0,1)=(-36,-36,-35)$.

	
	
\end{document}










