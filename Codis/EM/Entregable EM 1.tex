\documentclass[12pt, a4papre]{article}
\usepackage[catalan]{babel}
\usepackage[unicode]{hyperref}
\usepackage{amsmath}
\usepackage{amssymb}
\usepackage{amsthm}
\usepackage{xifthen}
\usepackage{array}

\newcommand{\norm}[1]{\lvert #1 \rvert}

\hypersetup{
    colorlinks = true,
    linkcolor = blue
}

\author{Daniel Vilardell}
\title{Entregable Electromagnetisme}
\date{}

\begin{document}
	\maketitle
	\section{}
	Les calcularem tenint en conte que:
	\[
		(s, \phi, z) = (x^2 + y^2, \arctan\left(\frac{y}{x}\right), z)
	\]
	\[
		(r, \theta, \phi) = (x^2 + y^2 + z^2, \arccos\left(\frac{z}{r}\right),  \arctan\left(\frac{y}{x}\right))
	\]
	\begin{center}
		{\renewcommand{\arraystretch}{1.3}
		\begin{tabular}{ |>{\centering\arraybackslash}p{0.3\textwidth} | >{\centering\arraybackslash}p{0.3\textwidth} | >{\centering\arraybackslash}p{0.3\textwidth} |}
			\hline
			$\textbf{(x, y, z)}$			& $\pmb{(s, \phi, z)}$			& $\pmb{(r, \theta, \phi)}$		\\ \hline
			(0, 3, 4)	 				& $(3, \frac{\pi}{2}, 4)$  		& $(5, 0.643, \frac{\pi}{2})$	\\ \hline
			(0, -3, 4)					& $(3, \frac{3\pi}{2}, 4)$ 		& $(5, 0.643, \frac{3\pi}{2})$		\\ \hline
			(-3, 0, 4) 					& $(3, \pi, 4)$ 				& $(5, 0.643, \pi)$				\\ \hline
			(3, 4, 6)					& (5, 0.927, 6) 				& $(7.81, 0.694, 0.927)$			\\ \hline
			(-3, 4, 6) 					& (5, -0.927, 6) 				& $(7.81, 0.694, -0.927)$			\\
			\hline
		\end{tabular}
		}
	\end{center}
	
	\section{}
	En primer lloc passarem la coordenada $(-3, 5, 4)$ amb les identitats vistes abans
	\[
		(s, \phi, z) = (5.830, 2.11, 4)
	\]
	I ara que tenim el vector en coordenades cilindriques podem calcular la expressio en coordenades tenint en conte que 
	\[
		\vec{\theta}=(-\sin(\theta), \cos(\theta), 0) = (-\sin(2.11), \cos(2.11), 0) = (0.857, -0.514, 0)
	\]
	
	\[
	\vec{B} = \frac{4\pi\cdot 10^{-7}\cdot 1}{2\pi 5.83}(0.85\vec{i} - 0.51\vec{j}) = (2.91\cdot10^{-8}, -1.75\cdot10^{-8}, 0)
	\]
	
	\section{}
	En primer lloc passem el punt P a coordenades esferiques per a despres trobar la matriu del canvi de base
	\[
		P = (r, \theta, \phi) = (4,\frac{\pi}{3},\frac{3\pi}{2})
	\]
	I a partir d'aqui apliquem la matriu de canvi de base
	\[
		\begin{pmatrix}
			A_x \\
			A_y \\
			A_z
		\end{pmatrix}
		= 
		\begin{pmatrix}
			\sin (\theta)\cos (\phi) 	&  \cos (\theta)\cos (\phi) 	& -\sin(\phi)\\
			\sin (\theta)\sin (\phi) 	& \cos (\theta)\sin (\phi)  	& \cos(\phi)\\ 
			\cos(\theta) 			&-\sin(\theta)			&0
		\end{pmatrix}
		\cdot
		\begin{pmatrix}
			A_r \\
			A_{\theta}\\
			A_{\phi}
		\end{pmatrix}
	\]
	\[
		\begin{pmatrix}
			A_x \\
			A_y \\
			A_z
		\end{pmatrix}
		=
		\begin{pmatrix}
			0				&0				&1	\\
			-\frac{\sqrt{3}}{2}	&-\frac{1}{2}		&0		\\
			\frac{1}{2}			&\frac{\sqrt{3}}{2}	&0
		\end{pmatrix}
		\begin{pmatrix}
			5\\
			4.33\\
			0
		\end{pmatrix}
		=
		\begin{pmatrix}
			0 \\
			-6.5 \\
			-1.25
		\end{pmatrix}
	\]
	\section{}
	\textbf{a)}
	Amb les identitats vistes en el exercici 1 es veu que 
	\[
		P = (r, \theta, \phi) = (4, \frac{\pi}{6}, \pi) 
	\]
	\textbf{b)}
	Altre cop amb les identitats vistes en el exercici 1 es veu que
	\[
		\vec{A} = \left(\frac{2\cos(\frac{\pi}{6})}{4^3}, 0, \frac{sin(\frac{\pi}{6})}{4^3}\right) = (0.0270, 0, 0.0078)
	\]
	\[
		\vec{A} = (x,y,z) = (0,0,0.0270)
	\]

	
	
	\section{}
	
	\begin{center}
		$$Q = \int_{0.03}^{0.05} -\frac{3\cdot10^{-8}}{r^4}4\pi r^2 dx = \int_{0.03}^{0.05} -\frac{12\cdot10^{-8}\pi}{r^2} dx=
		-12\cdot10^{-8}\pi\int_{0.03}^{0.05} \frac{1}{r^2} dx = $$
		$$ =12\cdot10^{-8}\pi \left[ \frac{1}{r}  \right]_{0.03}^{0.05} = -5.0265\cdot 10^{-6} C$$
		
	\end{center}
	
\end{document}










