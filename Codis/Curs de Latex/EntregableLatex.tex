\documentclass[12pt, a4papre]{article}
\usepackage[catalan]{babel}
\usepackage[unicode]{hyperref}
\usepackage{amsmath}
\usepackage{amssymb}
\usepackage{amsthm}
\usepackage{xifthen}

\hypersetup{
    colorlinks = true,
    linkcolor = blue
}

\author{Daniel Vilardell}
\title{Sessio 2 del millor curs}
\date{}

\newtheoremstyle{theoremdd}% name of the style to be used
  {\topsep}% measure of space to leave above the theorem. E.g.: 3pt
  {\topsep}% measure of space to leave below the theorem. E.g.: 3pt
  { }% name of font to use in the body of the theorem
  {0pt}% measure of space to indent
  {\bfseries}% name of head font
  {\\}% punctuation between head and body
  { }% space after theorem head; " " = normal interword space
  {\Large \thmname{#1}{ 3.}}

\theoremstyle{theoremdd}
\newtheorem{thmd}{Example}[section]


\newcommand{\mathset}[2][]{\ifthenelse{\equal{#1}{}}{\ifthenelse{\equal{#2}{}}{$\emptyset$}{$\left\{#2 \right\}$}}{$\left\{\left.#1 \right\vert  #2\right\}$}}

\begin{document}
\setcounter{section}{3}
\maketitle

\begin{thmd}
 Solve the PDE
 \begin{center}
 	\begin{equation}
		\boxed{u_x + 2xy^2u_y=0}
		\label{eq1}
	\end{equation}
 \end{center}
 The characteristic curves satisfy the ODE $dy/dx = 2xy^2/1=2xy^2$. To solve the ODE, we separate variables:
  $dy/y^2=2x dx$; hense $-1/y = x^2-C$, so that
  
 \begin{center} 
 	$y = (C-x^2)^{-1}$.
\end{center}

 These curves are the characteristics. Again, $u(x,y)$ is a constant on each such curve.m (Check it by writing it out.) So
 $u(x,y)=f(C)$, where $f$ is an arbitrary function. Theredore, the general solution of  is obtained by solving \eqref{eq1} 
 is obtained by solving \eqref{eq2} for $C$. That is,
 
 \begin{center}
 	\begin{equation}
		\boxed{u(x,y)=f\left(x^2+\frac{1}{y}\right).}
		\label{eq2}
	\end{equation}
 \end{center}
 
 Again this is easily checked by differentiation, using the chain rule:  \newline $u_x=2xf'(x^2+\frac{1}{y})$ and
  $u_y=-(\frac{1}{y^2}f'(x^2+\frac{1}{y})$, whence $u_x+2xy^2u_y=0$.
\qed
\end{thmd}

\mathset[\frac{x}{1+\frac{x}{1+\frac{x}{\frac{x}{e}}}}]{x \in \mathbb{R}}\
,
\mathset{}\
,
\mathset{ \sum_{3}}\

\mathset[(x,y)\in \mathbb{R}^2]{x\in A}

\end{document}