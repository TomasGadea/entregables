\documentclass[12pt, a4papre]{article}
\usepackage[catalan]{babel}
\usepackage[unicode]{hyperref}
\usepackage{amsmath}
\usepackage{amssymb}
\usepackage{amsthm}
\usepackage{xifthen}
\usepackage{setspace}

\newcommand{\norm}[1]{\lvert #1 \rvert}

\hypersetup{
    colorlinks = true,
    linkcolor = blue
}

\author{SKAPIENS}
\title{JIRAFA}
\date{}

\begin{document}

	\maketitle
	\begin{center}
		Que viva darwin!!!\\
		Skapiens se queda en casa\\
		
		\begin{doublespacing}
		(jirafa x3) \\
		\end{doublespacing}
		(jirafa x2 ska)\\
		
		\begin{doublespacing}
		Hoy os voy a presentar,\\
		\end{doublespacing}
		Un animal peculiar\\
		Ella no sabe caminar\\
		Solo puede bailar ska\\
		
		\begin{doublespacing}
		Marrón y amarillo su color\\
		\end{doublespacing}
		Ella a Jamaica se marchó\\
		
		\begin{doublespacing}
		Es una jirafa ska es una jirafa ska\\
		\end{doublespacing}
		Jirafa ska jirafa ska\\
		Jirafa ska jirafa\\
		
		\begin{doublespacing}
		Esa jirafa descubrió\\
		\end{doublespacing}
		Un zoológico\\
		Era muy trágico\\
		Bastante ilógico\\
		
		\begin{doublespacing}
		Ella se quiso vengar\\
		\end{doublespacing}
		Del homo sapiens vulgar\\
		
		\begin{doublespacing}
		Hay demasiada evolucion en contra de la razon\\
		\end{doublespacing}
		Y esa jirafa tenia la solución\\
		Entabló conversación con el reggae león\\
		Y llegaron a la gran conclusión\\
		
		\[
		\begin{pmatrix}
		\textrm{Que el humano es un animal}\\
		\textrm{(a prisión)}\\
		\textrm{A ver si le hace gracia ser una distracción}
		\end{pmatrix}^4
		\]
		
		\begin{doublespacing}
		Animal confinado dime como se siente\\
		\end{doublespacing}
		Estas en el zoo sin animales de público\\
		Aminal confinado dime como se siente\\
		Estas en el zoo homo sapiens inconsciente\\
		
		

	\end{center}

\end{document}